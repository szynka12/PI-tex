\chapter{Projekt}
    \section{Wstęp}
    Niniejszy raport prezentuje wyniki prac nad projektem mechanicznego zegara wieżowego.
    W ramach projektu wykonano:
   \begin{itemize}
   	\item model 3D oraz dokumentację płaską w programie NX,
   	\item obliczenia dynamiki mechanizmu zegarowego,
   	\item symulacje wytrzymałościowe najbardziej wytężonych podzespołów.
   \end{itemize}
	
	
        \subsection{Motywacja}
        Powodem podjęcia tego właśnie tematu w ramach przedmiotu Projekt Integrujący, było uświetnienie budynku Instytutu Techniki Cieplnej.
        Ponadto dodatkową pobudką była chęć zmierzenia się z nietrywialnym problemem zaprojektowania analogowego zegara o dużych gabarytach.
        Jest to skomplikowany mechanizm, którego sposób działania nie jest zbyt dobrze udokumentowany w powszechnie dostępnej współczesnej literaturze. Takie zegary są zwykle projektowane w sposób rzemieślniczy. Kolejną motywacją była chęć odkrycia metodologii projektowania takich mechanizmów od strony inżynierskiej.
        \subsection{Przegląd rozwiązań technicznych w technice zegarowej}
        Wśród większości mechanicznych zegarów wieżowych można znaleźć wiele podobieństw.
        Działanie najważniejszych części może być jednak zrealizowane w różny sposób:
        \begin{enumerate}
        	\item Układ taktujący - istnieją dwa rozwiązania wychwytowe:
        	\begin{itemize}
        		\item symetryczna kotwica wychwytowa - jest zdecydowanie trudniejsza w wykonaniu, wymaga dokładnych tolerancji i bardzo dobrego dopasowania kształtów zębów koła wychwytowego i kotwicy. Jest wykorzystywana w małych dokładnych zegarach.
        		\item niesymetryczna kotwica wychwytowa - jest prostsza w wykonaniu, lecz odmierza czas z mniejszą dokładnością. Jest zwykle wykorzystywana w zegarach wieżowych gdzie dokładność co do sekundy nie jest zwykle potrzebna i osiągalna.
        	\end{itemize}
        \item Mechanizm magazynowania energii:
			\begin{itemize}
				\item sprężyna skrętna - wykorzystywana głownie w zegarkach naręcznych, wymaga bardzo lekkiego mechanizmu o małych oporach ruchu, a więc też dużej dokładności wykonania.
				\item masa na linie nawinięta na bęben - rozwiązanie wykorzystywane w zegarach stojących (tzw. \textit{grandfather clock}) oraz wieżowych.
			\end{itemize}
       	\end{enumerate}	
		
    \section{Założenia projektowe}
    Na początku tworzenia projektu przeanalizowano wiele aspektów całego przedsięwzięcia.
    Miało to na celu stworzyć pierwszy konceptu tego projektu, zdefiniować istotne wielkości dotyczące samej konstrukcji jak i eksploatacji oraz nadać im wstępne wartości.
    W ten sposób powstały wymienione niżej założenia projektowe:
    \begin{itemize}
    	\item Średnica tarczy zegara: 2m. 
    	Wymiar ten dobrano na podstawie obserwacji budynku, na którego szczycie ma się znaleźć zegar.
    	Stwierdzono, że taka średnica pozwoli przechodniom dostrzec wyraźnie pokazywaną przez zegar godzinę.
    	\item Długość wahadła: 4 - 10m.
    	Wybrano taki przedział, ponieważ z jednej strony wahadło ma być widoczne i tym samym pełnić rolę estetyczną konstrukcji.
    	Z drugiej zaś strony nie może być zbyt długie, gdyż rodziłoby to problemy konstrukcyjne.
    	\item Gabaryty mechanizmu: 1x1x1m (z wyłączeniem wahadła).
    	\item Okres drgań pracy wahadła: 4s.
    	Przede wszystkim istotne jest aby była to liczba naturalna, aby w dalszym etapie można było łatwiej dobrać koła zębate przekładni.
    	\item Czas pracy zegara między nakręceniami: 8 dni.
    	Przy określeniu tej wielkości brano pod uwagę możliwe wystąpienie skumulowanych wolnych dni od pracy.
    	
    \end{itemize}

    \section{Podział konstrukcji na moduły}
        \subsection{Układ napędowy - przekładnia mechanizmu}
           \subsubsection{Ciąg główny}
           		Ciąg główny odpowiada łączy magazyn energii z układem taktującym.
           		Układ taktujący naprzemiennie blokuje i pozwala na ruch całego mechanizmu.
           		Kiedy przekładnia jest odblokowana moment siły jest przekazywany przez zestaw przekładni do koła wychwytowego i do wskazówek, połączonych z ciągiem głównym poprzez mechanizm wskazowy.
           		Ciąg główny składa się z 4 kół zębatych o zarysie ewolwentowym, z koła wychwytowego oraz z 4 zębników połączonych z poszczególnymi kołami.
  				Każdy z zębników ma 12 zębów.
  				
  				Układ składa się z następujących elementów (patrz rysunek \ref{fig:kacpermechanizmassmdwg1}):
  				\begin{description}
  					\item[Koło główne] - jest ono połączone z magazynem energii zegara. Posiada ono 96 zębów.
  					\item[Zębnik nr 1] - współpracuje z kołem głównym.
  					\item[Koło nr 1] - jest połączone z zębnikiem nr 1. Posiada 72 zęby.
  					\item[Zębnik nr 2] - współpracuje z kołem nr 1.
  					\item[Koło nr 2] - jest połączone z zębnikiem nr 2. Posiada 60 zębów.
  					\item[Zębnik nr 3] - współpracuje z kołem nr 2.
  					\item[Koło nr 3] - jest połączone z zębnikem nr 3. Posiada 30 zębów.
  					\item[Zebnik nr 4] - współpracuje z kołem nr 3.
  					\item[Koło wychwytowe] - jest połączone z zębnikiem nr 4. Posiada 36 niesymetrycznych zębów.
  				\end{description}
  				Razem koła zapewniają przełożenie 600:1.
  				Biorąc pod uwagę, że koło wychwytowe wykonuje 1200 obrotów na jeden dzień pracy (koło ma 36 zębów; mechanizm taktujący pozwala na przesunięcie o jeden ząb na 2 sekundy, czyli pół założonego okresu wahadła) przełożenia zostały dobrane w ten sposób aby otrzymać odpowiednie prędkości obrotowe wskazówek.
  				Koło główne wykonuje 2 obroty w ciągu jednego dnia.
  				\begin{figure}
  					\centering
  					\includegraphics[width=\linewidth]{Projekt/ciag_glowny}
  					\caption{Ciąg główny mechanizmu napędowego.}
  					\label{fig:kacpermechanizmassmdwg1}
  				\end{figure}
  				
  			\subsubsection{Mechanizm wskazowy} 
  				
  				Mechanizm wskazowy składa się z trzech kół zębatych oraz dwóch zębników (patrz rysunek \ref{fig:mechwskaz}):
  				\begin{description}
  					\item[Koło centralne] - współpracuje z kołem nr 1. Posiada 48 zębów.
  					\item[Zebnik nr 5] - jest połączony z kołem centralnym. Posiada 15 zębów.
  					\item[Koło godzinowe nr 1] - współpracuje z zębnikiem nr 5. Posiada 45 zębów. 
  					\item[Zębnik nr 6] - jest połączony z kołem godzinowym nr 1. Posiada 12 zębów.
  					\item[Koło godzinowe nr 2] - współpracuje z zębnikiem nr 6. Posiada 48 zębów. 
  				\end{description} 
  				Koło centralne jest jednocześnie kołem połączonym ze wskazówką minutową. Koła godzinowe zapewniają razem przełożenie 1:12.
  				Dzięki temu wskazówka godzinowa obraca się 2 razy w ciągu jednego dnia.
  				
  				\begin{figure}
  					\centering
  					\includegraphics[width=\linewidth]{Projekt/mech_wskaz}
  					\caption{Mechanizm wskazowy}
  					\label{fig:mechwskaz}
  				\end{figure}
  				
        \subsection{Układ napędowy - magazyn energii}
			Zegar napędzany jest mechanicznie, za pomocą masy na linie nawiniętej na bęben, który jest połączony przez sprzęgło jednokierunkowe z przekładnią. W celu poprawnego działania zegara, napęd musi dysponować wystarczającym momentem, który w każdym takcie dostarczy energię na:

\begin{itemize}
\item Dołożenie energii do wahadła
\item Skompensowanie strat ciernych w przekładni
\item Skompensowanie oporów stawianych wskazówkom przez wiatr oraz ptaki na nich siadające.
\end{itemize}

Powyższe źródła strat stanowią podstawę do budowy modelu energetycznego mechanizmu, na podstawie którego wyznaczono wymiary i masy elementów napędu.

Jako dane wejściowe przyjęto następujące wielkości, przedstawione w tabeli \ref{tab:bebn_zal}:

\begin{table}[h]
\centering
\begin{tabular}{l|c|c|c}
Nazwa & Symbol & Wartość & Jednostka \\ \hline \hline
Czas pracy pomiędzy nawijaniem& \(T_z\) & 8 & dni \\
Okres wahadła & \(T_w\) & 4 & s \\
Prędkość kątowa bębna & \(\omega_b\) & 2 & obr/dzień\\
Przełożenie & \(i_z\) &  600  & - \\
Sprawność przekładni & \(\eta_z\) & 0.8 & - \\
Sprawność wahadła w 1 cyklu & \(\eta_w\) & 0.995 & -\\
Amplituda kątowa wahadła & \(\theta_w\) & 5 & stopnie \\
Moment bezwładności wahadła & \(I_w\) & 1 & \(kg m^2\) \\
Długość bezwładności wahadła & \(L_w\) & 1 & m \\
Masa wahadła & \(m_w\) & 300 & kg \\
\hline
\end{tabular}
\caption{Założenia do pracy zegara. Wszystkie obliczenia prowadzone są w jednostkach SI.}
\label{tab:bebn_zal}
\end{table}


Znając wymiary wahadła można wyznaczyć ilość energii w nim skumulowanej
\begin{equation}
E_w = m_w g  L_w (1-cos(\theta_w))
\end{equation}

Ze sprawności można wyznaczyć stratę energii w jednym cyklu:

\begin{equation}
\Delta E_w = (1-\eta_w)E_w,
\end{equation}

oraz potrzebną ilość energii jaką musi dostarczyć układ napędowy w każdym takcie:

\begin{equation}
\Delta E_b = \frac{\Delta E_w}{\eta_z}.
\end{equation}

Założony czas pracy i częstotliwość pracy wahadła wyznacza ilość potrzebnych cykli:

\begin{equation}
n_z = \lfloor\frac{T_z}{T_w}\rfloor
\end{equation}

Które wyznaczają ilość energii jaką musi wydzielić układ napędowy w jednym okresie pracy (pomiędzy nakręcaniem) \(T_z\):
\begin{equation}
E_b = n_Z \Delta E_b
\end{equation}

Podobnie można wyznaczyć całkowity obrót bębna:
\begin{equation}
\Delta \Theta = \frac{T_z}{\omega_b}
\end{equation}

Parametry swobodne w projekcie bębna to średnica bębna \(D_b\) oraz masa ciężarka \(m_c\). W celu jak najmniejszego obciążenia konstrukcji, Należy tak dobrać te parametry aby moment generowany przez bęben był jak najmniejszy. Można go wyznaczyć bardzo prosto:
\begin{equation}
M_b = \frac{D_b}{2}m_c g
\end{equation}

Jednak dobór nie jest dowolny. Istnieją ograniczenia konstrukcyjne:

Ograniczeniem na średnicę bębna jest maksymalna wysokość o jaką może zostać opuszczony ciężarek - innymi słowy jest to wysokość na której jest zamontowany zegar.

\begin{equation}
\label{eqn:Dbbound}
D_b \leq \frac{\Delta h_{cmax}}{\Delta \Theta_b}
\end{equation}

Zależnie od doboru długości liny i wstępnego nawinięcia, ciężarek może osiąść na ziemi, lub - w przypadku rozwinięcia się całej liny - ramię momentu będzie zerowe. W obydwu przypadkach zegar przestanie pracować, gdyż zatrzymana przekładnia bardzo szybko wytraci energię zmagazynowaną w wahadle.

Ponadto drugie ograniczenie wynika z energii zmagazynowanej w wiszącym ciężarku, które w czasie jednego okresu pracy nie może być mniejsza niż wymagana przez cały mechanizm:
\begin{equation}
\label{eqn:EbBound}
\Delta \Theta_b \frac{D_b}{2} m_c g \geq E_b
\end{equation}

Ostatecznie postawiony jest problem optymalizacji z ograniczeniami. W tym prostym przypadku, można go rozwiązać analitycznie. Ograniczenie \ref{eqn:Dbbound} jest zależne jedynie od średnicy, zatem można zamienić nierówność na równość:

\begin{equation}
D_b = \frac{\Delta h_{cmax}}{\Delta \Theta_b}
\end{equation}

Masa ciężarka jest zatem wyznaczona z ograniczenia \ref{eqn:EbBound}, również poprzez przyjęcie równości:
\begin{equation}
m_c = \frac{2E_b}{g\Delta \Theta_b D_b}
\end{equation}

Takie rozwiązanie daje moment równy:
\begin{equation}
M_b = \frac{E_b}{\Delta \Theta_b}
\end{equation}
Powyższy wzór nie przypadkiem przypomina zależność na pracę którą musi wykonać moment w jednym okresie pracy.

Powyższe wielkości uwzględniają niepewności projektowe, wyrażone przez sprawności przekładni i wahadła. Sprawność przekładni uwzględnia również straty generowane przez wskazówki zegara obciążone warunkami atmosferycznymi i gołębiami siedzącymi na wskazówkach, które wprowadzają moment oporowy. Należyty dobór momentu, bez zbędnego naddatku jest istotny, gdyż wszelkie niespożytkowane zasoby zostaną wydzielone na kole wychwytowym, co będzie prowadzić do szybszego zużycia mechanizmu.

Powyższe wyniki posłużą do dalszych obliczeń, wytrzymałościowych bębna i wahadła.
\begin{figure}
	\centering
	\includegraphics[width=0.9\linewidth]{Projekt/beben}
	\caption{Magazyn energii wahadła - bęben na który nawinięta jest lina z ciężarkiem.}
	\label{fig:beben}
\end{figure}

		\subsection{Osadzenie kół na wałach}

		W celu osadzenia kół na wale zdecydowano się na połączenia wciskowe. Zadecydowało o tym kilka powodów:
		\begin{enumerate}
			\item Jest dobrą i o wiele tańszą alternatywą dla wykonywania części pełniącej jednocześnie funkjcę zębnika i wału.
			\item Połączenie jest nierozłączne dzięki czemu ma bardzo niskie wymagania jeśli chodzi o konserwację (brak ruchomych elementów takich jak np. wpust). 
			\item Jest dobrą alternatywą dla spawania, ponieważ wał potrafi wykrzywić się podczas spawania. Połączenie wciskowe jest dodatkowo prostsze technologicznie (wymaga tylko podgrzania osadzanej części, w przypadku spawania wymagana jest dodatkowa obróbka po połaćzeniu obu części).
			\item Zębnik jest małym elementem, co utrudnia zastosowanie wpustów.
			\item Jest to metoda typowo spotykana w zegarach.
		\end{enumerate}
		Dokonano obliczeń najbardziej obciążonych połączeń wciskowych - koła głównego z wałem i zębnika nr 1 z tym samym wałem.

		\subsubsection{Koło główne}
		\paragraph{Parametry zadane}
		\begin{itemize}
			\item maksymalny moment obrotowy $M = 60$ [Nm]
			\item średnica wału $d_{wal} = d = 30$ [mm]
			\item szerokość połączenia $b = 15$ [mm]
			\item średnica okręgu bazowego koła $d_2 = 70$ [mm]
			\item średnica wewnętrzna wału $d_{wal}^{wew} = d_1 = 0$ [mm]
			\item współczynnik nadmiaru nośności $k_{mom} = 1.3$ [-]
			\item materiał wału - S355jr, $Re = 330$ [MPa],
			\item materiał koła - C55, hartowana powierzchniowo, $Re = 540s$ [MPa],
			\item współczynnik tarcia $f_{w-k} = 0.14$ - łączenie ogrzaniem-oziębieniem.
		\end{itemize}
		Obliczono:
		\begin{enumerate}
			\item Wielkość współczynników:
				\begin{align}
					c_1 &= \dfrac{(d^2 +d_1^2)}{(d^2 - d_1^2)} - \upsilon_1 = 0.7 \\[2em]	
					c_2 &= \dfrac{(d_2^2 +d^2)}{(d_2^2 - d^2)} + \upsilon_2 = 1.75
				\end{align}
				
				przy czym $\upsilon_1 = \upsilon_2 = 0.3$ - współczynniki Poissona dla wału i koła.
				
			\item Minimalny wcisk na powierzchni styku
				\begin{align}
					p_{min} &\geq \dfrac{2 k_{mom} T \cdot 10^3}{\pi d^2 b f_{w-k}} = 26.27 \ \text{[MPa]} 
				\end{align}
			
			\item Minimalny wcisk obliczeniowy w połączeniu
				\begin{align}
					N_{min} &= 10^3 \cdot p_{min} d \left(\dfrac{c_1}{E_1} + \dfrac{c_2}{E_2}\right) = 9.41 \ [\mu \text{m}] 
				\end{align}
			
				przy czym $E_1 = E_2 = 205$ [GPa] - moduł Younga materiałów.
				
			\item Korekta na plastyczne odkształcenie wierzchołków nierówności powierzchni podczas montażu \\
			Przyjęto chropowatości czopa i otworu krzywki równe $R_{z1} = R_{z2} = 6.3$ - szlifowanie dokładne.
				\begin{align}
					\gamma = 1.2 (R_{z1} + R_{z2}) = 15.12 \ [\mu \text{m}]
				\end{align}
			
			\item Minimalny dopuszczalny wcisk w połączeniu
				\begin{align}
					N_{min \ dop} = N_{min} + \gamma = 24.53 \ [\mu \text{m}] 
				\end{align}
			
			
			\item Maksymalny nacisk na powierzchni styku
			\begin{itemize}
				\item dla wału:
				\begin{align}
				P_{1 \ max} &= 0.5 Re \left[1 - \left(\dfrac{d_1}{d}\right)^2\right] = 270 \ \text{[MPa]}
				\end{align}
				
				\item dla koła:
				\begin{align}
				P_{2 \ max} &= 0.5 Re \left[1 - \left(\dfrac{d}{d_2}\right)^2\right] = 220,40 \ \text{[MPa]}
				\end{align}
			\end{itemize}

			\item Największy obliczeniowy wcisk w połączeniu
			\begin{align}
				N_{max} &= 10^3 \cdot p_{max (1,2)} d \left(\dfrac{c_1}{E_1} + \dfrac{c_2}{E_2}\right) = 96,80 \ [\mu \text{m}] 
			\end{align}

			\item Największy dopuszczalny wcisk w połączeniu
			\begin{align}
				N_{max \ dop} = N_{max} + \gamma = 111,92 \ [\mu \text{m}]
			\end{align}
			\end{enumerate}
			
			\par Dobrano następujące pasowanie:
			\begin{enumerate}
				\item Rodzaj pasowania \tab $H6/t6$
				
				\item Górna odchyłka otworu	\tab $ES = +13 \ [\mu \text{m}]$
				\item Dolna odchyłka otworu \tab $EI = 0 \ [\mu \text{m}]$
				\item Górna odchyłka wałka \tab $es = +54 \ [\mu \text{m}]$
				\item Dolna odchyłka wałka \tab $ei = +41 \ [\mu \text{m}]$
				\item Maksymalny wcisk \tab $N_{max} = es - EI = 54 \ [\mu \text{m}]$
				\item Minimalny wcisk \tab $N_{min} = ei - ES = 28 \ [\mu \text{m}]$ \\
			\end{enumerate}
		% 
		
		% \end{enumerate}
		
		\noindent Warunek wytrzymałości części jest spełniony:
		\begin{align}
			N_{max} &\leq N_{max \ dop} 
		\end{align}
		
		\noindent Warunek wytrzymałości połączenia jest spełniony:
		\begin{align}
			N_{min} &\geq N_{min \ dop} 
		\end{align}
		
		\noindent Różnica temperatur do montażu skurczowego:
		\begin{align}
			\Delta T &= \dfrac{N_{max}}{d \cdot \alpha_T} \approx 163 \ \text{[K]}
		\end{align}
		
		gdzie $\alpha_T \approx 11 \cdot 10^{-6} \ \left[\dfrac{1}{\text{K}}\right]$ to współczynnik rozszerzalności cieplnej stali.

		\subsubsection{Zębnik}
		\paragraph{Parametry zadane}
		\begin{itemize}
			\item maksymalny moment obrotowy $M = 7,5$ [Nm]
			\item średnica wału $d_{wal} = d = 20$ [mm]
			\item szerokość połączenia $b = 30$ [mm]
			\item średnica okręgu bazowego koła $d_2 = 28,5$ [mm]
			\item średnica wewnętrzna wału $d_{wal}^{wew} = d_1 = 0$ [mm]
			\item współczynnik nadmiaru nośności $k_{mom} = 1.3$ [-]
			\item materiał wału - S355jr, $Re = 360$ [MPa],
			\item materiał zębnika - C55, hartowana powierzchniowo, $Re = 540s$ [MPa],
			\item współczynnik tarcia $f_{w-k} = 0.14$ - łączenie ogrzaniem-oziębieniem
		\end{itemize}
		Obliczono:
		\begin{enumerate}
			\item Wielkość współczynników:
				\begin{align}
					c_1 &= \dfrac{(d^2 +d_1^2)}{(d^2 - d_1^2)} - \upsilon_1 = 0.7 \\[2em]	
					c_2 &= \dfrac{(d_2^2 +d^2)}{(d_2^2 - d^2)} + \upsilon_2 = 3,24
				\end{align}
				
				przy czym $\upsilon_1 = \upsilon_2 = 0.3$ - współczynniki Poissona dla wału i zebnika.
				
			\item Minimalny wcisk na powierzchni styku
				\begin{align}
					p_{min} &\geq \dfrac{2 k_{mom} T \cdot 10^3}{\pi d^2 b f_{w-k}} = 3,69 \ \text{[MPa]} 
				\end{align}
			
			\item Minimalny wcisk obliczeniowy w połączeniu
				\begin{align}
					N_{min} &= 10^3 \cdot p_{min} d \left(\dfrac{c_1}{E_1} + \dfrac{c_2}{E_2}\right) = 1,42 \ [\mu \text{m}] 
				\end{align}
			
				przy czym $E_1 = E_2 = 205$ [GPa] - moduł Younga materiałów.
				
			\item Korekta na plastyczne odkształcenie wierzchołków nierówności powierzchni podczas montażu \\
			Przyjęto chropowatości czopa i otworu krzywki równe $R_{z1} = R_{z2} = 6,3$ - szlifowanie dokładne.
				\begin{align}
					\gamma = 1.2 (R_{z1} + R_{z2}) = 15.12 \ [\mu \text{m}]
				\end{align}
			
			\item Minimalny dopuszczalny wcisk w połączeniu
				\begin{align}
					N_{min \ dop} = N_{min} + \gamma = 16,54 \ [\mu \text{m}] 
				\end{align}
			
			
			\item Maksymalny nacisk na powierzchni styku
			\begin{itemize}
				\item dla wału:
				\begin{align}
				P_{1 \ max} &= 0.5 Re \left[1 - \left(\dfrac{d_1}{d}\right)^2\right] = 255 \ \text{[MPa]}
				\end{align}
				
				\item dla koła:
				\begin{align}
				P_{2 \ max} &= 0.5 Re \left[1 - \left(\dfrac{d}{d_2}\right)^2\right] = 114,19 \ \text{[MPa]}
				\end{align}
			\end{itemize}

			\item Największy obliczeniowy wcisk w połączeniu
			\begin{align}
				N_{max} &= 10^3 \cdot p_{max (1,2)} d \left(\dfrac{c_1}{E_1} + \dfrac{c_2}{E_2}\right) = 86,5 \ [\mu \text{m}] 
			\end{align}

			\item Największy dopuszczalny wcisk w połączeniu
			\begin{align}
				N_{max \ dop} = N_{max} + \gamma = 101,62 \ [\mu \text{m}]
			\end{align}
			\end{enumerate}
			
			\par Dobrano następujące pasowanie:
			\begin{enumerate}
				\item Rodzaj pasowania \tab $H6/s6$
				\item Górna odchyłka otworu	\tab $ES = +13 \ [\mu \text{m}]$
				\item Dolna odchyłka otworu \tab $EI = 0 \ [\mu \text{m}]$
				\item Górna odchyłka wałka \tab $es = +56 \ [\mu \text{m}]$
				\item Dolna odchyłka wałka \tab $ei = +35 \ [\mu \text{m}]$
				\item Maksymalny wcisk \tab $N_{max} = es - EI = 56 \ [\mu \text{m}]$
				\item Minimalny wcisk \tab $N_{min} = ei - ES = 22 \ [\mu \text{m}]$ \\
			\end{enumerate}
		% 
		
		% \end{enumerate}
		
		\noindent Warunek wytrzymałości części jest spełniony:
		\begin{align}
			N_{max} &\leq N_{max \ dop} 
		\end{align}
		
		\noindent Warunek wytrzymałości połączenia jest spełniony:
		\begin{align}
			N_{min} &\geq N_{min \ dop} 
		\end{align}
		
		\noindent Różnica temperatur do montażu skurczowego:
		\begin{align}
			\Delta T &= \dfrac{N_{max}}{d \cdot \alpha_T} \approx 254,5 \ \text{[K]}
		\end{align}
		

        % \subsection{Wskazówki}
        % Kolejnym istotnym elementem projektu są wskazówki.
        % Może się wydawać, że istotny jest tylko ich wygląd.
        % Jednakże ze względu na obecne w układzie przełożenie, muszą być wystarczająco dobrze wyważone.
        % W innym wypadku moment siły wprowadzany z bębna może się okazać nie wystarczający.
        
        \subsection{Układ nastawny}
        Niezbędnym elementem każdego zegara jest moduł pozwalający skorygować wskazywaną godzinę.
        Wiąże się to między innymi ze skończoną dokładnością wykonania wszystkich elementów zegara oraz konieczność zatrzymania go na czas serwisu.
        Analizując możliwe rozwiązania oraz istniejące już konstrukcje, postanowiono zaprojektować ten układ z użyciem sprzęgła.
        Ma ono następujące zalety:
        \begin{itemize}
        	\item Pozwala na dowolny obrót ustawianymi wskazówkami.
        	\item Nie wymaga zatrzymania układu taktującego na czas ustawiania godziny.
        \end{itemize}
		Piasta sprzęgła działa w sposób zaciskowy (patrz rysunek \ref{fig:szpeglo}).
		
		\begin{figure}
			\centering
			\includegraphics[width=0.8\linewidth]{Projekt/szpeglo}
			\caption{Mechanizm ustawiania godziny.}
			\label{fig:szpeglo}
		\end{figure}
		
    	
        \subsection{Układ taktujący}
            Układ taktujący składa się z trzech głównych elementów:
            \begin{itemize}
                \item wahadła;
                \item koła wychwytowego;
                \item wychwytu.
            \end{itemize}
            Okres drgań wahadła fizycznego jest stały. Dzięki temu możliwe jest wykorzystanie wahadła do regulacji czasu jaki zajmuje jeden takt zegara.
            Ruch wahadła jest sprzęgnięty z cała resztą mechanizmu za pomocą koła zębatego nazywanego wychwytowym oraz wychwytu.
            
            Wychwyt jest elementem w kształcie kotwicy (rysunek~\ref{fig::wychwyt}), który naprzemiennie blokuje~i pozwala na swobodny ruch koła wychwytowego.
            Koło jest połączone poprzez przekładnie z całą resztą układu napędowego.
            W ten sposób, gdy koło wychwytowe jest zablokowane, cały mechanizm stoi w miejscu, natomiast gdy może ono się ruszać, moment jest przekazywany przez cały układ napędowy~i wskazówki się przesuwają.

            Ruch wychwytu jest sprzęgnięty z ruchem wahadła. W ten sposób, dobierając odpowiednio ilość zębów koła wychwytowego oraz przełożenia w układzie napędowym, można skorelować długość okresu wahadła z ilością taktów zegara.
            
            \begin{figure}[th]
            \centering
            \includegraphics[width=0.9\linewidth]{Obliczenia/wychwyt}
            \caption{Schematyczny rysunek mechanizmu wychwytowego. Na rysunku widoczne są dwa stany w jakich może znajdować się mechanizm. Po lewej stronie zaprezentowana jest sytuacja, w której ruch koła jest zablokowany przez wychwyt po lewej stronie. Po prawej, koło wychwytowe zostało zwolnione, obróciło się (stare położenie jest zaznaczone szarym kolorem), a następnie zostało zablokowane po prawej stronie. Strzałką zaznaczono kierunek obrotu koła wychwytowego.} 
            \label{fig::wychwyt}
            \end{figure}
			
			\begin{figure}[th]
				\centering
				\includegraphics[width=0.6\linewidth]{Projekt/takt}
				\caption{Mechanizm taktujący}
				\label{fig:takt}
			\end{figure}
			
		\subsection{Sworzeń wahadła}
		
		Wahadło zawieszone zostanie na sworzniu średnicy \(D_S = 30 mm\), wykonanym ze stali A2. Jej granica plastyczności to \(Rm = 500\). Sworzeń jest podparty w dwóch punktach oddalonych od siebie o \(L = 120 mm\). Ciężar wahadła wynosi \( m_w =50 kg\). Maksymalny moment gnący w sworzniu wynosi
		
		\begin{equation}
		Mg = \frac{m_w g L}{2} = \frac{50 \cdot 10 \cdot 0.12}{2} = 30\; Nm
		\end{equation}
			
Moment bezwładności przekroju kołowego to \(I_s = \frac{\pi D_s^4}{64}\). Maksymalne naprężenia w sworzniu wynoszą:

\begin{equation}
\sigma_{\max} = \frac{M_g D}{2Ix} \approx 5.8\; MPa
\end{equation}

Granica zmęczenia na zginanie może zostać przybliżona jako \(Z_g = 0.5Rm\). Podczas pracy wahadło, poza obciążeniem statycznym przekazuje do sworznia również obciążenia dynamiczne, pochodzące od sił odśrodkowych. Przy amplitudzie \(A = 4^o\), i okresie \(T = 4 \; s\) maksymalna prędkość na końcu wahadła o długości \(L_w = 4\;m\) wynosi:
\begin{equation}
v_{\max} = \frac{A \cdot 2\pi}{T}L = \frac{4 \cdot 2\pi}{360 \cdot 4}\cdot4 = 0.3\; \dfrac{\text{m}}{\text{s}}
\end{equation}
Co odpowiada sile odśrodkowej
\begin{equation}
F_w = \frac{m v_{\max}^2}{L} = 35\; N
\end{equation}
Siła ta przekłada się na zwiększone naprężenia w sworzniu
\begin{equation}
\sigma_{\max} = 10\; MPa
\end{equation}
Współczynnik bezpieczeństwa zmęczeniowego na zginanie wynosi zatem:
\begin{equation}
\delta_s = \frac{Z_g}{\sigma_{\max}} = \frac{250}{10} = 25
\end{equation}
Co zapewnia wytrzymałość zmęczeniową i doraźną.

        \subsection{Rama}
        	Rama zostanie wykonana ze spawanych kształtowników stalowych. Podstawowym założeniem było aby była ona niezależna od całej reszty mechanizmu. Wszystkie koła zębate są zamocowane na dwóch stalowych płytach przykręconych do kształtowników. Bęben, na który nawinięta jest lina z ciężarkiem jest ułożyskowany na dwóch łożyskach wahliwych tak, aby kompensować niedokładności ramy wynikające ze spawania.
        	
        	
        \subsection{Projektowe obliczanie wałów}
        	Wszystkie dobrane koła zębate oraz układ dostarczania energii wymagają dobrania wałów o odpowiednich średnicach.
        	W tym celu przeprowadzono obliczenia zginania i skręcania dwóch najbardziej obciążonych wałów: wału do którego przymocowany jest bęben z opuszczający obciążenie oraz wału koła głównego.
        	
        	Jako pierwszy został poddany analizie wał do którego przymocowany jest bęben.
        	Na rysunku \ref{fig::shaft_beben} przedstawiono przyłożone obciążenia (przypadek rozwijania liny z ciężarem, na kolejnym przedstawiony jest drugi przypadek) oraz umiejscowienie łożysk. Siły $F_1$ i $F_2$ pochodzą od ciężaru bębna wraz ciężarem dostarczającym energię.
        	Zaś siła $F_3$ to ciężar wolnobiegu.
        	Z kolei $M_skr$ pokazuje gdzie wprowadzony jest moment skręcający od bębna oraz gdzie odebrany(wolnobieg).
        	Wartości obciążeń oraz wymiarów umieszczono w tabeli \ref{tab::beben} dla 3 położeń obciążenia na bębnie.
        	
        	
        
        
        	\begin{figure}[th]
        		\centering
        		\includegraphics[width=0.9\linewidth]{Projekt/shaft_beben}
        		\caption{Schemat obciążeń przyłożonych do wału, na którym zamocowany jest wolnobieg oraz bęben podczas opuszczania ciężaru zamocowanego na bębnie} 
        		\label{fig::shaft_beben}
        	\end{figure}
        	\begin{figure}[th]
				\centering
				\includegraphics[width=0.9\linewidth]{Projekt/shaft_beben2}
				\caption{Schemat obciążeń przyłożonych do wału, na którym zamocowany jest wolnobieg oraz bęben podczas nawijania na bęben liny z ciężarem} 
				\label{fig::shaft_beben2}
			\end{figure}        
        
        	W przypadku wału z bębnem konieczne jest rozważenie kilku przypadków:
        \begin{itemize}
        	\item Opuszczanie ciężarka - w tym wypadku moment skręcający jest przekazywany od bębna w strone wolnobiegu.
        	\item Nakręcanie zegara - w tym wypadku moment skręcający jest przekazywany od korby do bębna.
        	\item Położenie ciężarka na jednym z końców lub w środku bębna - przypadki te należy sprawdzić niezależnie dla obu powyższych.
        \end{itemize}
        
       		Reasumując, należy przeanalizować 6 przypadków obciążeń. 
       		
       		\begin{table}[h]
       			\centering
       			\begin{tabular}{l|c|c|c}
       				Wielkość & Początek & Środek & Koniec \\ \hline \hline
       				$F_1$ [N]& 1299,83 & 711,225 & 122,625 \\ 
       				$F_2$ [N]& 122,625 & 711,225 & 1299,83 \\ 
       				$F_3$ [N] & \multicolumn{3}{|c|}{98,1}\\
       				$M_{skr}$ [Nmm] &  \multicolumn{3}{|c|}{60 000} \\
       				$L$ [mm] & \multicolumn{3}{|c|}{304} \\
       				$L_1$ [mm] & \multicolumn{3}{|c|}{72}\\
       				$L_2$ [mm]  & \multicolumn{3}{|c|}{242} \\
       				$L_3$ [mm]  & \multicolumn{3}{|c|}{382} \\
       				
       				\hline
       			\end{tabular}
       			\caption{Wartości obciążeń i wymiarów wałka z bębnem.}
       			\label{tab::beben}
       		\end{table}
        	
        	\begin{figure}[th]
        		\centering
        		\includegraphics[width=0.9\linewidth]{Projekt/momenty_gnace_rozwijanie}
        		\caption{Wykresy momentów gnących dla 3 przypadków umiejscowienia obciążenia na bębnie.} 
        		\label{fig::momenty_gnace_rozwijanie}
        	\end{figure}
        
